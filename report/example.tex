% #################################################
\newcommand{\titel}{Vorlage für wissenschaftliche Arbeiten}
\newcommand{\untertitel}{Untertitel dieser Arbeit}
\newcommand{\art}{Masterarbeit}
\newcommand{\fachgebiet}{Elektro- und Informationstechnik}
\newcommand{\autor}{Max Beispiel}
\newcommand{\studienbereich}{Elektro- und Informationstechnik}
\newcommand{\matrikelnr}{123\,456\,78}
\newcommand{\jahr}{\the\year}
\newcommand{\ort}{Ingolstadt}
\newcommand{\fachbereich}{Elektro- und Informationstechnik}
\newcommand{\erstpruefer}{Prof. Dr. Christian Pfitzner}
\newcommand{\zweitpruefer}{Prof. Dr. Christian Pfitzner}

\newcommand{\schlagwortA}{SchlagwortA}
\newcommand{\schlagwortB}{SchlagwortC}
\newcommand{\schlagwortC}{SchlagwortC}
\newcommand{\schlagwortD}{SchlagwortD}     

\newcommand{\zitat}{"Jede hinreichend fortgeschrittene Technologie \\ ist von Magie nicht mehr zu unterscheiden."\\[2ex]00
	\textit{Sir Arthur Charles Clarke  (1917-2008)}, \\
	 Britischer Science-Fiction Autor, Erfinder, and Futurist}
\newcommand{\vorwort}{Ich danke allen die mir geholfen haben. } %
\newcommand{\zusammenfassungDE}{Lorem ipsum dolor sit amet, consetetur sadipscing elitr, sed diam nonumy eirmod tempor invidunt ut labore et dolore magna aliquyam erat, sed diam voluptua.} %
\newcommand{\zusammenfassungEN}{Lorem ipsum dolor sit amet, consetetur sadipscing elitr, sed diam nonumy eirmod tempor invidunt ut labore et dolore magna aliquyam erat, sed diam voluptua.} %

\def\textLanguage{ger}      % ger oder eng
\def\print{true}	        % true oder false

% ################################################
\documentclass[
    12pt,										
    DIV12,
    ngerman, 							
    a4paper, 												    		
    %twoside,
    oneside,   								
    %openright,								
    onecolumn, 
    titlepage, 							
    parskip=half, 					
    headings=normal, 
    listof=totoc, 						
    bibliography=totoc, 			
    index=totoc, 	
    captions=tableheading, 	
    final 										
]{ohm_tex_report}


\definecolor{maincolor}{RGB}{188,199,46}



\graphicspath{{01_images/}}

% #################################################
% Document
% #################################################
\begin{document}



% auch subsubsection nummerieren
\setcounter{secnumdepth}{3}
\setcounter{tocdepth}{3}

% Deckblatt und Abstract ohne Seitenzahl
\ifoot{}
\ofoot{}
\thispagestyle{plain}
\begin{titlepage}

\begin{center}
\huge{\textbf{\titel}}\\[4.5ex]
\LARGE{\textbf{\art}}\\[1.5ex]
\Large{\ifx \textLanguage\eng 	in Department of \else im Fachgebiet \fi \fachgebiet}\\[4ex]

\includegraphics[scale=0.8]{02_images/00_LogoOhmHochschule_Blau.pdf}\\
\Large{\textbf{\hochschule}}\\[3ex]

\normalsize
\begin{tabular}{w{5.2cm}p{6.4cm}}\\
\ifx \textLanguage\eng presented by \else vorgelegt von \fi:  & \quad \autortitle \\
 							 & \quad \sc{\autor} \KleinerAbsatz
\ifx \textLanguage\eng Department \else Fachgebiet \fi: 		 & \quad Electrical Engineering and \\ 
							 & \quad Information Technology\KleinerAbsatz
\ifx \textLanguage\eng Course of studies: \else Studiengang: \fi & \quad Master of Applied Research \\
               & \quad in Engineering Sciences \KleinerAbsatz
\ifx \textLanguage\eng Matriculation number\else Matrikelnummer\fi: & \quad \matrikelnr \KleinerAbsatz
\ifx \textLanguage\eng Supervisor\else Betreuer\fi:  	 & \quad \betreuer\\[2ex]  					
\end{tabular}

\copyright\ \jahr\\[5.5ex]

\begin{table}[htbp]
	\begin{tabular}{ll}
		\textbf{\ifx \textLanguage\eng Key words \else Schlüsselwörter \fi}:
				& \schlagwortA, \schlagwortB, \schlagwortC, \schlagwortD
	\end{tabular}
\end{table}
\end{center}
\vfill
\singlespacing
\small
\noindent 
\ifx \textLanguage\eng
\small{
Declaration:
Herewith I declare that this thesis is the result of my independent work. All sources and auxiliary materials used by me in this thesis are cited completely.}
\else
\small{
Dieses Werk einschließlich seiner Teile ist \textbf{urheberrechtlich geschützt}. Jede Verwertung außerhalb der engen Grenzen des Urheberrechtgesetzes ist ohne Zustimmung des Autors unzulässig und strafbar. Das gilt insbesondere für Vervielfältigungen, Übersetzungen, Mikroverfilmungen sowie die Einspeicherung und Verarbeitung in elektronischen Systemen.}
\fi
\end{titlepage}

\textcolor{white}{Zitat}


\vfill
\begin{flushright}
\zitat
\end{flushright}
\newpage


\section*{Vorwort}
\vorwort

\newpage
\section*{Zusammenfassung}
\label{sec:Zusammenfassung}
\zusammenfassungDE


\vfill
\section*{Abstract}
\zusammenfassungEN







\ofoot{\pagemark}

\pagenumbering{Roman}	
\tableofcontents 		

% Abkürzungsverzeichnis --------------------------------------------------------
%\nomenclature[aA]{$\mu$}{Haftreibungskoeffizient}{}{}
\nomenclature[aA]{$\Phi$}{Haftreibungswinkel}{$^\circ$}{}
\nomenclature[aA]{$\alpha$}{Steigungswinkel}{$^\circ$}{}
\nomenclature[ab]{$b$}{Breite}{m}{L}%
\nomenclature[ae]{$e$}{Regelabweichung}{}{}%
\nomenclature[aF]{$\vecF{F}$}{Kraft}{N}{$\mathrm{M \cdot L \cdot T^{-2}}$}%
%\nomenclature[aG]{$g$}{Erdbeschleunigung}{$\mathrm{\tfrac{m}{s^2}}$}{$\mathrm{L \cdot T^{-2}}$}%
\nomenclature[ah]{$h$}{Höhe}{m}{L}%
\nomenclature[al]{$l$}{Länge}{m}{L}%
\nomenclature[ad]{$d$}{Distanz}{m}{L}%
\nomenclature[aM]{$M$}{Drehmoment}{$\mathrm{Nm}$}{$\mathrm{M \cdot L^2 \cdot T^{-2}}$}
%\nomenclature[am]{$m$}{Masse}{kg}{M}%
%\nomenclature[an]{$n$}{Drehzahl}{$\mathrm{s^{-1}}$}{$\mathrm{T^{-1}}$}%
\nomenclature[at]{$t$}{Zeit}{s}{T}%
\nomenclature[av]{$v$}{Geschwindigkeit}{$\mathrm{m \cdot s^{-1}}$}{$\mathrm{L\cdot T^{-1}}$}%
\nomenclature[ax]{$x$}{Position}{m}{L}%
\nomenclature[af]{$f$}{Brennweite}{m}{L}%
\nomenclature[af]{$f$}{Frequenz}{$\mathrm{s^{-1}}$}{$\mathrm{T^{-1}}$}%
%\nomenclature[as]{$s$}{Sensorwert}{}{}%
%\nomenclature[aU]{$U$}{Spannung}{V}{$\mathrm{M \cdot L^2 \cdot I^{-1} \cdot T^{-3}}$}%
\nomenclature[aS]{$S$}{Sensor}{}{}%
\nomenclature[aI]{$I$}{Strom}{A}{I}%

\nomenclature[aR]{$\mat{R}$}{3x3 Rotationsmatrix}{}{}
\nomenclature[aT]{$\mat{T}$}{4x4 Transformationsmatrix}{}{}
\nomenclature[aT]{$\mat{Tr}$}{4x4 Translationsmatrix}{}{}
\nomenclature[an]{$\vec{n}$}{Normalenvektor in $\mathbb{R}^3$}{}{}
\nomenclature[aP]{$\vec{p}$}{Punkt in $\mathbb{R}^3$ }{}{}
\nomenclature[ar]{$\vec{r}$}{Ortsvektor in $\mathbb{R}^3$}{}{}
\nomenclature[ae]{$\vec{e}$}{Einheitsvektor in $\mathbb{R}^3$}{}{}
\nomenclature[aK]{$K$}{Verstärkungsfaktor}{}{}
\nomenclature[aT]{$T$}{Zeitkonstante}{s}{T}
\nomenclature[aw]{$w$}{Winkelgeschwindigkeit}{$\mathrm{rad \cdot s^{-1}}$}{$\mathrm{T^{-1}}$}
\nomenclature[ag]{$\nabla \vec{g}$}{Gradientenvektor in $\mathbb{R}^2$}{}{}

%\nomenclature[zI]{$ist$}{Istwert}{}{}%
%\nomenclature[zm]{$max$}{Maximalwert}{}{}%
%\nomenclature[zm]{$min$}{Minimalwert}{}{}%
%\nomenclature[zS]{$soll$}{Sollwert}{}{}%
%\nomenclature[zd]{$default$}{Defaultwert}{}{}%
%\nomenclature[zn]{$n-1$}{vorangegangen}{}{}%
%\nomenclature[zn]{$n$}{Zählerwert}{}{}%
%\nomenclature[zc]{$c$}{Centroid}{}{}%


%\clearpage\markboth{\nomname}{\nomname}
%% Punkte zw. Abkürzung und Erklärung
%\setlength{\nomlabelwidth}{.45\hsize}
%\renewcommand{\nomlabel}[1]{#1 \dotfill} 
%\printnomenclature

%G:\00_Bachelor_latex>makeindex -s nomentbl.ist -o masterarbeit.nls masterarbeit.nlo

%\input{Verzeichnisse/Abkuerzungen}
\label{sec:Glossar}

\ifx \textLanguage\eng
	\addcontentsline{toc}{chapter}{List of Abbreviations}
\else
	\addcontentsline{toc}{chapter}{Abkürzungsverzeichnis}	
\fi
\listoffigures 	
\listoftables 	

% arabische Seitenzahlen im Hauptteil --------------------------------------\ifx \textLanguage\eng----
\clearpage
\pagenumbering{arabic} 
\spacing{1.3}
% #################################################
% Inhalt 
% #################################################
%\include{Inhalt/01_Einleitung}
%\include{Inhalt/content}



\chapter{Einleitung}
\section{Erster Untergliederungspunkt}

%\nocite{*}
\bibliographystyle{plain} 
%\inputencoding{latin2}
\bibliography{references}
%\inputencoding{utf8}

% #################################################
% Anhang 
% #################################################
\begin{appendix}
    \clearpage    
    \pagenumbering{roman}
    \setdefaultleftmargin{1em}{}{}{}{}{}     
    %\input{Anhang}
\end{appendix}


% \spacin{1.3}



\end{document}
